\documentclass[12pt]{article}

\usepackage{preamble}

\begin{document}

\newpage \maketitle

\newpage \tableofcontents

\newpage \newsection{Introduction}

\subsection{Section.} Introduction.

\newpage \newsection{The real numbers.}

\subsection{Section.} The real numbers.

\subsection{Definition.} Denote the natural numbers $\N$, the integers $\Z$, and the rational numbers $\Q$.

\subsection{Remark.} An axiomatic treatment of $\N, \Z, \Q$ is beyond the scope of this course.

\subsection{Remark.} For many purposes, the rationls are not "big enough".

\subsection{Example.} \label{ex:sqrt2} There is no $x \in \Q$ such that $x^{2} = 2$. Indeed, assume there are $p, q \in \Z$ coprime such that $(p/q)^{2} = 2$. Then $p^{2} = 2q^{2}$, which means that $p^{2}$ is even, so $p$ is even as well. But if $p$ is even, then $p^{2} = 2q^{2}$ is divisible by $4$. But this is only possible if $q$ is even as well, a contradiction of the coprime-ness of $p, q$.

\subsection{Example.} In fact, there is not even a "best approximation" in $\Q$ of the solution to $x^{2} = 2$. Consider the sets 
$$A = \{a \in \Q \mid a^{2} < 2, a > 0\} \text{ and } B = \{b \in \Q \mid b^{2} > 2, b > 0\}.$$
$A$ has no largest element. Indeed, suppose $p \in A$ with $p > 0$. Then setting 
$$q = \frac{2p + 2}{p + 2}$$
has that $q \in A$ and $q > p$. A similar argument shows that $B$ has no smallest element.

\subsection{Definition.} An order on a set $S$ is a relation $<$ such that 

(1) (Trichotomy) For $x, y \in S$, exactly one of the statements hold: 
$$x < y \text{ or } x = y \text{ or } y < x.$$

(2) (Transitivity) For $x, y, z \in S$, if $x < y$ and $y < z$, then $x < z$.

If a set $S$ is equipped with an order $<$, then $(S, <)$ or simply $S$ is called an ordered set.

\subsection{Example.} $\N, \Z, \Q$ are all ordered by $x < y$ iff $y - x$ is positive.

\subsection{Definition.} Let $S$ be an ordered set. $E \subseteq S$ is bounded above if there exists some $M \in S$ such that $x \leq B$ for all $x \in E$. Similarly, $E \subseteq S$ is bounded below if there exists some $L \in S$ such that $x \geq B$ for all $x \in E$.

\subsection{Example.} $\N$ has the well-ordering principle, in that every $E \subseteq \N$ nonempty has a least element.

\subsection{Definition.} Let $S$ be an ordered set and let $E$ be bounded above. If there exists an $\alpha$ such that 

(1) $\alpha$ is an upper bound of $E$,

(2) If $\beta < \alpha$, then $\beta$ is not an upper bound of $E$,

then $\alpha = \sup E$ is the supremum or least upper bound of $E$. Define symmetrically $\inf E$ to be the infimum or greatest lower bound of $E$.

\subsection{Example.} 

A supremum/infimum, when it exists, need not be a member of the subset.

(1) The sets $A, B \subseteq \Q$ defined in Example \ref{ex:sqrt2} have no supremum or infimum, respectively.

(2) $\N \subseteq \Z$ is bounded below but has no least upper bound.

(3) $E = \{1/n \mid n \in \Z_{+}\}$ has that $\inf E = 0 \not\in E$.

\subsection{Definition.} An ordered set has the least upper bound property if whenever $E \subseteq S$ is nonempty and bounded above, then $\sup E$ exists. Define symmetrically the greatest lower bound property.

\subsection{Example.} $\N, \Z$ have the greatest upper bound property. $\Q$ does not.

\subsection{Theorem.} Let $S$ be an ordered set with the least upper bound property and let $B \subseteq S$ is nonempty and bounded below. Let $L$ be the set of all lower bounds of $B$. Then $\sup L$ exists and is equal to $\inf B$, which exists.

\subsection{Proof.} $L \neq \emptyset$ since $B$ is bounded below. Since $y \leq x$ for every $y \in L, x \in B$, then every $x \in B$ is an upper bound of $L$. Thus $L$ is bounded above since $B$ is nonempty, so $\sup L$ exists by the least upper bound property. Since every $x \in B$ is an upper bound of $L$, $\sup L \leq x$ for every $x \in B$, so $\sup L$ is a lower bound of $B$. For any lower bound $y \in L$ of $B$, $y \leq \sup L$, so $\sup L = \inf B$.

\subsection{Remark.} A set $S$ has the least upper bound property iff it has the greatest lower bound property.

\subsection{Definition.} A field $(F, + , \cdot)$ is a set $F$ equipped with two binary operations $+: F \times F \to F$ and $\cdot: F \times F \to F$ that satisfy the field axioms: For any $x, y, z \in F$,

(1) (Closure) $x + y \in F$ and $x \cdot y \in F$.

(2) (Commutativity) $x + y = y + x$ and $x \cdot y = y \cdot x$.

(3) (Associativity) $(x + y) + z = x + (y + z)$ and $(x \cdot y) \cdot z = x \cdot (y \cdot z)$.

(4) (Identity) There exists two symbols $0, 1 \in F$ with $0 \neq 1$ such that $0 + x = x$ and $1 \cdot x = x$.

(5) (Inverse) There exists $-x \in F$ such that $x + (-x) = 0$ for any $x$, and there exists $1/x \in F$ such that $x \cdot 1/x = 1$ for any $x \neq 0$.

(6) (Distibutivity) $x \cdot (y + z) = xy + xz$.

\subsection{Example.}

(1) $\Q$ with $+$ and $\cdot$ defined normally as a field.

(2) $\{0, 1\}$ is the trivial field.

(3) $\Z$ is not a field because it does not have multiplicative inverses.

\subsection{Remark.} Fields are interesting because any statement proven about a general field $F$ must hold in any field such as $\Q, \R, \C$ (to be defined later).

\subsection{Definition.} An ordered field is a field $F$ equipped with an order $<$ such that for $x, y, z \in F$, 

(1) $x + y < x + z$ if $y < z$.

(2) $xy > 0$ if $x, y > 0$.

$x > 0$ is said to be positive and $x < 0$ is said to be negative.

\subsection{Theorem.} There exists an ordered field $\R$ with the least upper bound property. Moreover, $\Q \subseteq \R$. The elements of $\R$ are called the real numbers.

\subsection{Proof.} The proof is delayed until the end of the course (though the tools already exist to prove it).

\subsection{Theorem.} $\R$ has the follow properties:

(1) (Archimedean property) Lorem ipsum.

(2) (Density of $\Q$) Lorem ipsum.

(3) (Existence of roots) Lorem ipsum.

\subsection{Proof.}

Lorem ipsum.

\newpage \newsection{Sequences and series of functions}

\subsection{Section.} Sequences and series of functions.

\subsection{Definition.} Given a sequence of $(f_{n})$ of $\C$-valued functions on a metric space $(X, d)$ such that $\lim f_{n}(x)$ exists for every $x \in X$, then define the limit $\lim f_{n}$ to be the function $f: X \to \C$ such that 
$$f(x) = \lim_{n \to \infty} f_{n}(x)$$
for every $x \in X$. $(f_{n})$ is said to converge pointwise to $f$

\subsection{Remark.} Do limits/sums of functions preserve the properties of the sequence? If $(f_{n})$ is a sequence of continuous/differentiable functions, then is the limit/sum continuous/differentiable? Moreover, is $(f_{n}')$ related to $f'$?

Recall that $f$ is continuous at $x$ iff $f(t) \to f(x)$ as $t \to x$. Thus, asking whether the limit of continuous functions is continuous is the same as asking if 
$$\lim_{t \to x} \lim_{n \to \infty} f_{n}(t) = \lim_{n \to \infty} \lim_{t \to x} f_{n}(t),$$
namely if it is possible to "swap limits".

\subsection{Example.} Pointwise convergence is not sufficient to swap limits. Let $S_{m,n} = m/(m + n)$ for each $m, n \in Z_{+}$. Then, 
$$\lim_{n \to \infty} \lim_{m \to \infty} S_{m,n} = \lim_{n \to \infty} 1 = 1$$
while 
$$\lim_{m \to \infty} \lim_{n \to \infty} S_{m, n} = \lim_{m \to \infty} 0 = 0.$$

\subsection{Example.} Pointwise convergence is not enough to guarantee continuity of limits! For $x \in \R$, let 
$$f(x) = \sum_{n = 0}^{\infty} f_{n}(x) = \sum_{n = 0}^{\infty} \frac{x^{2}}{(1 + x^{2})^{n}}.$$
Since $f_{n}(0) = 0$ for every $n \in \N$, $f(0) = 0$. If $x \neq 0$, then 
$$f(x) = x^{2} + \frac{x^{2}}{1 + x^{2}} + \frac{x^{2}}{(1 + x^{2})^{2}} + \ldots = 1 + x^{2},$$
so 
$$f(x) = \begin{cases}
    0 & \text{if } x = 0 \\
    1 + x^{2} & \text{otherwise.}
\end{cases},$$
so $f$ is not continuous!

\subsection{Example.} For $x \in \R$ and $n \in \N$, let 
$$g_{n}(x) = \frac{\sin (nx)}{\sqrt{n}}$$
so that 
$$g(x) = \lim_{n \to \infty} g_{n}(x) = 0$$
for every $x \in R$. But 
$$g_{n}' = \sqrt{n} \cos (nx)$$
is such that 
$$\lim_{n \to \infty} g_{n}'(0) = \sqrt{n},$$
so $g_{n}'$ does not converge pointwise to $g'$.

\subsection{Definition.} Let $(f_{n})$ be a sequence of $\C$-valued functions on a metric space $(X, d)$. $(f_{n})$ converges uniformly on $E \subseteq X$ if for every $\epsilon > 0$, there is an $N \in \N$ such that $n \geq N$ implies that 
$$|f_{n}(x) - f(x)| < \epsilon$$
for every $x \in E$. 

\subsection{Remark.} Uniform convergence is stronger than pointwise convergence: $f_{n} \to f$ uniformly implies that $f_{n} \to f$ pointwise.

\subsection{Theorem.} Let $(f_{n})$ be a sequence of $\C$-valued functions on a metric space $(X, d)$. $(f_{n})$ converges uniformly on $E \subseteq X$ iff $(f_{n})$ is uniformly Cauchy on $E$, namely iff for every $\epsilon > 0$ there is an $N \in \N$ such that $m, n \geq N$ implies that 
$$|f_{m}(x) - f_{n}(x)| < \epsilon$$
for every $x \in E$.

\subsection{Proof.} 

($\Rightarrow$) If $f_{n} \to f$ uniformly on $E$ and $\epsilon > 0$, then there is an $N \in \N$ such that $n \geq n$ implies that $|f_{n}(x) - f(x)| < \epsilon/2$ for every $x \in E$. Hence, for $m, n \geq N$, 
$$|f_{m}(x) - f_{n}(x)| \leq |f_{m}(x) - f(x)| + |f(x) - f_{n}(x)| < \epsilon/2 + \epsilon/2$$
for every $x \in E$, so $(f_{n})$ is uniformly Cauchy in $E$.

($\Leftarrow$) If $(f_{n})$ is uniformly Cauchy on $E$ and $\epsilon > 0$, then there is an $N \in |N$ such that $m, n \geq n$ implies that $|f_{m} - f_{n}| < \epsilon/2$ for every $x \in E$. The sequence $(f_{m}(x))$ is Cauchy in $\C$ for every $x \in E$, so 
$$f(x) = \lim_{m \to \infty} f_{m}(x)$$
exists for every $x \in E$. This means that if $m, n \geq N$, then for each $x \in E$, 
$$|f_{m}(x) - f_{n}(x)| < \epsilon/2 \quad \Longrightarrow \quad -\epsilon/2 < f_{m}(x) - f_{n}(x) < \epsilon/2,$$
so that 
$$-\epsilon/2 \leq \lim_{m \to \infty} (f_{m}(x) - f_{n}(x)) = f(x) - f_{n}(x) \leq \epsilon/2,$$
which means that 
$$|f(x) - f_{n}(x)| < \epsilon$$
for every $x \in E$, so $f_{n} \to f$ uniformly on $E$.

\subsection{Theorem.} Suppose that $f_{n} \to f$ pointwise on $E \subseteq X$. Then $f_{n} \to f$ uniformly iff $f_{n} \to f$ in the supremum norm, namely, if 
$$\sup_{x \in E} |f_{n}(x) - f| \to 0$$
as $n \to \infty$.

\subsection{Proof.}

($\Rightarrow$) If $f_{n} \to f$ uniformly then $\sup |f_{n}(x) - f(x)| \to 0$ by definition.

($\Leftarrow$) If $\sup |f_{n}(x) - f(x)| \to 0$, then for every $\epsilon > 0$, there is an $N \in \N$ such that $n \geq N$ implies that $\sup |f_{n}(x) - f(x)| < \epsilon$. Hence, for $n \geq N$, 
$$|f_{n}(x) - f(x)| \leq \sup_{x \in E} |f_{n}(x) - f(x)| < \epsilon$$
for every $x \in E$, so $f_{n} \to f$ uniformly.

\subsection{Example.} The sequence of functions $f_{n}(x) = 1/(nx + 1)$ on $(0, 1) \supseteq \R$ for $n \in \N$ is such that $f_{n} \to 0$ pointwise but for every $n$, 
$$|0 - f_{n}(x)| = \left| \frac{1}{nx+1} \right|,$$
so choosing $x = 1/n \in (0, 1)$ has that 
$$\left| 0 - f_{n}\left(\frac{1}{n}\right) \right| = \frac{1}{2}$$
for every $n \in \N$. So $f_{n}$ does not converge to $0$ uniformly.

\subsection{Theorem.} (Weierstrauss M-test) If $(f_{n})$ is a sequence of $\C$-valued functions on $E \subseteq X$ for a metric space $(X, d)$ with $|f_{n}(x)| \leq M_{n}$ for every $x \in E, n \in \N$, then $\sum f_{n}$ converges uniformly if $\sum M_{n}$ converges.

\subsection{Proof.} If $\sum M_{n}$ converges in $\R$, then its partial sums are Cauchy in $\R$. Hence, for every $\epsilon > 0$, there is an $N \in \N$ such that $m, n \geq N$ implies by the triangle equality that 
$$\left| \sum_{i=m}^{n} f_{i}(x) \right| \leq \sum_{i=m}^{n} |f_{i}(x)| \leq \sum_{i=m}^{n} M_{i} < \epsilon$$
so that the partial sums of $\sum f_{n}$ are uniformly Cauchy and hence uniformly convergent.

\subsection{Example.} The converse statement to the Weierstrauss M-test fails in general. Choose the sequence of functions on $\R$ defined by 
$$f_{n}(x) = \frac{(-1)^{n+1}}{n}$$
for $n \in Z_{+}$. Then 
$$\sum_{n=1}^{\infty} f_{n}(x) = \log 2$$
for every $x \in \R$. But $|f_{n}(x)| = 1/n$ for every $x \in X$, so setting $M_{n} = 1/n$ has that $\sum M_{n}$ diverges. Hence $\sum f_{n}$ converges uniformly but $\sum M_{n}$ diverges, so the converse fails.

\subsection{Theorem.} Suppose that $f_{n} \to f$ uniformly on $E \subseteq X$ for a metric space $(X, d)$ and that $x \in X$ is a limit point of $E$ and that $f_{n}(t) \to A_{n}$ as $t \to x$ for every $n \in \N$. Then $(A_{n})$ converges and 
$$\lim_{t \to x} f(t) = \lim_{t \to x} \lim_{n \to \infty} f_{n}(t) = \lim_{n \to \infty} \lim_{t \to x} f_{n}(t) = \lim_{n \to \infty} A_{n}.$$
Moreover, if $(f_{n})$ is a sequence of continuous functions, then $f$ is continuous also.

\subsection{Proof.} Since $f_{n} \to f$ uniformly, the sequence is uniformly Cauchy, so for every $\epsilon > 0$, there is an $N \in \N$ such that $m, n \geq N$ implies that $|f_{m}(t) - f_{n}(t)| < \epsilon/2$ for every $t \in E$. Sending $t \to x$, $m, n \geq N$ implies that 
$$|A_{m} - A_{n}| \leq \epsilon/2 < \epsilon,$$
so $(A_{n})$ is uniformly Cauchy in $\C$ and thus converges to $A \in \C$. Note that for every $n \in \N$ and $t \in E$, 
$$|f(t) - A| \leq |f(t) - f_{n}(t) + |f_{n}(t) - A_{n}| + |A_{n} - A|$$
by the triangle inequality. For $\epsilon > 0$, choose $N \ni \N$ such that for $n \geq N$, both 
$$|f(t) - f_{n}(t)| < \epsilon/3$$
for every $t \in E$ (since $f_{n} \to f$ uniformly) and 
$$|A_{n} - A| < \epsilon/3$$
since $A_{n} \to A$. Finally, choose some neighborhood $U$ of $x$ in $X$ such that 
$$|f_{n}(t) - A_{n}| < \epsilon/3$$
for all $t \in (U \cap E) \setminus \{x\}$. Combining these three inequalities, 
$$|f(t) - A| < \epsilon,$$
for every $t \in (U \cap E) \setminus \{x\}$, to get the desired conclusions.

\subsection{Theorem.} (Dini's) Suppose that $(f_{n})$ be a sequence of $\R$-valued functions on a compact subset $K \subseteq X$ for a metric space $(X ,d)$. If $f_{n} \to f$ pointwise, $f_{n}$ continuous for every $n \in \N$, $f$ is continuos, and $f_{n} \geq f_{n+1}$ or $\leq$ for every $n \in \N$, then $f_{n} \to f$ uniformly on $K$.

\subsection{Proof.} Consider the sequence $(g_{n}) = (f_{n} - f)$. Then $g_{n}$ is continuous for every $n \in N$ and $g_{n} \to 0$ pointwise and $g_{n} \geq g_{n+1}$ for every $n \in \N$. For $\epsilon > 0$, let 
$$K_{n} = g_{n}^{-1}([\epsilon, \infty))$$
(which is closed since $g_{n}$ is continuous) for every $n \in \N$. Since $K$ is compact, $K_{n} \subseteq K$ is compact also. If $x \in K_{n+1}$, then $\epsilon \leq g_{n+1}(x) \leq g_{n}(x)$, so $x \in K_{n}$ also. Thus the $K_{n+1} \subseteq K_{n}$ are nested for $n \in N$. Now for each $x \in K$, $g_{n}(x) \to 0$ so that $x \not\in K_{n}$ eventually. Thus $\cap K_{n} = \emptyset$, which means that $K_{n} = \emptyset$ eventually since the $K_{n}$ are compact and nested. This means that there is some $N \in \N$ such that $K_{n} = \emptyset$ for all $n \geq N$, so $g_{n} < \epsilon$ for all $n \geq N$ for all $x \in K$. Since $g_{n} \geq 0$ for every $n \geq 1$, this implies that $g_{n} \to 0$ uniformly which implies that $f_{n} \to f$ uniformly.

\subsection{Remark.}

(1) Compactness is necessary for Dini's Theorem as exhibited by the sequence $(f_{n}(x)) = (1/(nx + 1))$ on $(0, 1)$ with $f_{n} \geq f_{n+1}$ for every $n \in \N$.

(2) Monotonacity is necessary for Dini's Theorem as exhibited by the example of the sequence $(g_{n})$ on $[0, 1]$ defined by 
$$g_{n}(x) = \begin{cases}
    nx & {if } 0 \leq x \leq 1/n \\
    2 - nx & \text{if } 1/n \leq x \leq 2/n \\
    0 \text{ otherwise.}
\end{cases}$$
Here, $g_{n} \to 0$ pointwise, but $g_{n}$ does not converge to $0$ uniformly.

\subsection{Theorem.} The set of bounded continuous functions $\mathcal{C}(X)$ on a metric space $(X, d)$ under the supremum norm is a complete metric space.

\subsection{Proof.} $(f_{n}) \subseteq \mathcal{C}(X)$ is Cauchy iff it is uniformly Cauchy iff there is some $f: X \to \C$ such that $f_{n} \to f$ uniformly on $X$. Since $(f_{n})$ is a sequence of continuous functions, then $f$ is continuous. Moreover, $f$ is bounded because $f_{n} \to f$ uniformly implies that $|f_{n}(x) - f(x)| < 1$ for every $x \in X$ eventually. So $f \in \mathcal{C}(X)$ and since $f_{n} \to f$ uniformly, $d(f_{n}, f) \to 0$ as $n \to \infty$.

\subsection{Theorem.} Let $(f_{n})$ be a sequence of $\C$-valued functions differentiable on $[a, b]$ such that $f_{n}(x_{0})$ converges for some $x_{0} \in [a, b]$. If $(f_{n}')$ converges uniformly on $[a, b]$, then $(f_{n})$ converges uniformly on $[a, b]$ to a function $f$ and 
$$f'(x) = \lim_{n \to \infty} f_{n}'(x)$$
for every $x \in [a, b]$.

\subsection{Proof.} Let $\epsilon > 0$. Since $(f_{n}(x_{0}))$ converges, it is Cauchy, so there is some $N \in \N$ such that 
$m, n \geq N$ implies that 
$$|f_{m}(x_{0}) - f_{n}(x_{0})| < \epsilon/2$$
and since $(f'_{n})$ is uniformly convergent, it is uniformly Cauchy, so taking $N$ potentially larger, $m, n \geq N$ implies that 
$$|f_{m}'(t) - f_{n}'(t)| < \frac{1}{2}\frac{\epsilon}{b-a}$$
for every $t \in [a, b]$. By the Mean Value Theorem, for any $x, t \in [a, b]$ there is some $\xi \in (x, t)$ (or $\xi \in (t, x))$ such that if $m, n \geq N$, then 
\[|(f_{m}(x) - f_{n}(x)) - (f_{m}(t) - f_{n}(t))| = |x-t||f_{m}'(\xi) - f_{n}'(\xi)| \leq \frac{1}{2}\frac{|x-t|}{b-a} \leq \frac{\epsilon}{2} \tag{$\ast$}\]
Hence for any $x \in [a, b]$, for $m, n \geq N$, 
$$|f_{m}(x) - f_{n}(x) \leq |(f_{m}(x) - f_{n}(x)) - (f_{m}{x_{0}} - f_{n}(x_{0}))| + |f_{m}(x_{0}) - f_{n}(x_{0})| < \epsilon/2 + \epsilon/2$$
so that $(f_{n})$ is uniformly Cauchy and hence uniformly convergent on $[a, b]$. Let $f$ be the limit of $(f_{n})$ and fix $x \in [a, b]$. Define for $t \in [a, b] \setminus \{x\}$ the functions 
$$\phi_{n}(t) = \frac{f_{n}(t) - f_{n}(x)}{t - x} \text{ and } \phi(t) = \frac{f(t) - f(x)}{t - x}.$$
Note then that 
$$\lim_{t \to x} \phi_{n}(t) = f_{n}'(x)$$
for every $n \in \N$. By ($\ast$), 
$$|\phi_{m}(t) - \phi_{m}(x)| \leq \frac{1}{2}\frac{\epsilon}{b-a}$$
for $m, n \geq \N$, so $(\phi_{n})$ converges uniformly on $[a, b] \setminus \{x\}$ as it is uniformly Cauchy. As $f_{n} \to f$ uniformly, $\phi_{n} \to \phi$ on $[a, b] \setminus \{x\}$, and as $x$ is a limit point of $[a, b] \setminus \{x\}$, 
$$\lim_{t \to x} \phi(t) = \lim_{t \to x} \lim_{n \to \infty} \phi_{n}(t) = \lim_{n \to \infty} \lim_{t \to x} \phi_{n}(t) = \lim_{n \to \infty} f_{n}'(x)$$
so that 
$$f'(x) = \lim_{t \to x} \phi(t) = \lim_{n \to \infty} f_{n}'(x)$$
for every $x \in [a, b]$.

\subsection{Example.} If $(f_{n}(x_{0}))$ converging for some $x_{0} \in [a, b]$ is not given, then $(f_{n})$ may not even converge. Consider the $(f_{n}) = n$ on $[0, 1]$ such that $f_{n}' = 0$ for every $n \in \N$ but $f_{n}$ diverges.

\subsection{Theorem.} There is a continuous function $f: \R \to \R$ that is nowhere differentiable.

\subsection{Proof.} Lorem ipsum.

\subsection{Definition.} A sequence $(f_{n})$ of $\C$-valued functions on a metric space $(X, d)$ is 

(1) pointwise bounded on $X$ if there is some $\phi: X \to \R$ such that $|f_{n}| \leq \phi$ for all $n \in \N$.

(2) uniformly bounded on $X$ if there is some $M \geq 0$ such that $|f_{n}| \leq M$ for all $n \in \N$.

\subsection{Remark.} Uniformly convergent implies uniformly bounded.

\subsection{Remark.} If $X$ is countable, then one can use the Cantor diagonalization argument to find a subsequence converging pointwise on $X$ if the sequence if pointwise bounded.

\subsection{Remark.} If $(f_{n})$ is uniformly bounded, it does not necessarily contain a pointwise convergent subsequence. This is shown in Rudin using the dominated convergence theorem on the sequence $(\sin (nx))$ on $[0, 2\pi]$.

\subsection{Example.} Convergent uniformly bounded sequences do not necessarily contain uniformly convergent subsequences. Let $(f_{n})$ be defined by 
$$f_{n}(x) = \frac{x^{2}}{x^{2} + (1 - nx)^{2}}$$
on $[0, 1]$ for $n \in \N$. Then $|f_{n}(x)| \leq 1$ so $(f_{n})$ is uniformly bounded. Moreover, $f_{n}(x) \to 0$ for every $x \in [0, 1]$. But $f_{n}(1/n) = 1$ for every $n \in \N$, so no subsequence converges uniformly.

\subsection{Theorem.} If $(f_{n})$ is a pointwise bounded sequence of $\C$-valued functions on a countable metric spcae $(X, d)$, then there is a subsequence $(f_{n_{k}})$ that converges for every $x \in X$.

\subsection{Proof.} Enumerate $X = (x_{i})$. As $(f_{n}(x_{1}))$ is bounded, there is a subsequence $(f_{k}^{1})$ of $(f_{n})$ such that $f_{k}^{1}(x_{1})$ converges as $k \to \infty$. As $f_{k}^{1}(x_{2})$ is bounded, there is a subsequence $(f_{k}^{2})$ of $(f_{k}^{1})$ such that $(f_{k}^{2}(x_{2}))$ converges as $k \to \infty$. Continuing inductively, consider the subsequence $(f_{k}^{l+1})$ of $f_{k}^{l}$ such that $(f_{k}^{l+1})(x_{l+1})$ converges as $k \to \infty$. Choose the diagonal subsequence $(f_{k}^{k})$ of $(f_{n})$ such that $(f_{k}^{k})(x_{i})$ converges for every $i \in \N$ since $(f_{k}^{k})$ is a subsequence of $(f_{k}^{i})$ for $k \geq i$ by construction. Reindexing $(f_{k}^{k}) = (f_{n_{k}})$ for every $k \in \N$ gives the desired subsequence.

\begin{center} \begin{tabular}{c|cccc}
    $x_{1}$ & $f_{1}^{1}$ & $f_{2}^{1}$ & $f_{3}^{1}$ & \ldots \\
    $x_{2}$ & $f_{1}^{2}$ & $f_{2}^{2}$ & $f_{3}^{2}$ & \ldots \\
    $x_{3}$ & $f_{1}^{3}$ & $f_{2}^{3}$ & $f_{3}^{3}$ & \ldots \\
    \vdots & \vdots & \vdots & \vdots
\end{tabular} \end{center}
The $i$th row converges for $x_{i}$. Every row is a subsequence of the row above. The diagonal subsequence is such that $(f_{k}^{k}(x_{i}))$ is convergent for every $i \in \N$.

\subsection{Definition.} A collection $\mathcal{F}$ of $\C$-valued functions on a set $E \subseteq X$ for a metric space $(X, d)$ is (uniformly) equicontinuous on $E$ if for every $\epsilon > 0$, there is some $\delta > 0$ such that for any $x, y \in E$, $d(x, y) < \delta$ implies that 
$$|f(x) - f(y)| < \epsilon$$
for every $f \in \mathcal{F}$.

\subsection{Remark.} If $\mathcal{F}$ is equicontinuous, then every $f \in \mathcal{F}$ is uniformly continuous.

\subsection{Example.}

(1) If $\mathcal{F} = (f_{n})$, a sequence of differentiable functions on $[0, 1]$ with $(f_{n}')$ uniformly bounded, then $\mathcal{F}$ is equicontinuous. Indeed, if $|f_{n}'(x)| \leq M$ for all $n \in \N$ and $x \in [0, 1]$, then for $\epsilon > 0$, choose $\delta = \epsilon / (M+1)$ so that by the Mean Value Theorem, 
$$|x - y| < \frac{\epsilon}{M+1}$$
implies that 
$$|f_{n}(x) - f_{n}(y)| \leq |x - y| \sup_{t \in [0, 1]} |f'(t)| < \frac{\epsilon M}{M+1} < \epsilon.$$

(2) The sequence 
$$\mathcal{G} = \left(\frac{x^{2}}{x^{2} + (1-nx)^{2}}\right)$$
on $[0, 1]$ is uniformly bounded, converges pointwise to $0$, but has no uniformly convergent subsequence. $\mathcal{G}$ is not equicontinuous as $g_{n}(1/n) = 1$ but $g_{n} = 0$ for every $n \in \N$, so there is no $\delta > 0$ for $\epsilon \in (0, 1)$.

(3) $\mathcal{H} = (\arctan (nx))$ is not equicontinuous since $\arctan (nx) \to \pm \pi/2$ if $x >/< 0$.

\subsection{Theorem.} Let $(K, d)$ be compact and $(f_{n}) \subseteq \mathcal{C}(K)$. Then if $(f_{n})$ converges uniformly on $K$, then $(f_{n})$ is equicontinuous on $K$.

\end{document}